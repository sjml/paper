\documentclass[
    12pt,
    letterpaper,
    oneside,
    noraggedright
]{turabian-researchpaper}

% borrows liberally from the default pandoc template;
% latest version here: https://github.com/jgm/pandoc/blob/main/data/templates/default.latex
% (maybe should just use that more?
%  handy to use the turabian-researchpaper package, but maybe
%  fighting it too much...)


% condition usage
\usepackage{ifthen}


\usepackage{fontspec}
\defaultfontfeatures{Scale=MatchLowercase}
\defaultfontfeatures[\rmfamily]{Ligatures=TeX,Scale=1}

% Tempora is a clone of Times Roman that
%   includes Greek and Cyrillic; should do
%   for most CSTM needs, but may need a Hebrew
%   solution some day.
\usepackage{tempora}
\setmainfont{Tempora-Regular.otf}[
    BoldFont       = Tempora-Bold.otf,
    ItalicFont     = Tempora-Italic.otf,
    BoldItalicFont = Tempora-BoldItalic.otf
]
\usepackage{sourcecodepro}
\setmonofont{SourceCodePro-Regular.otf}[
    BoldFont       = SourceCodePro-Bold.otf,
    ItalicFont     = SourceCodePro-RegularIt.otf,
    BoldItalicFont = SourceCodePro-BoldIt.otf
]

% macro for small-caps "LORD" even though Tempora
%   doesn't have proper small caps
%   Make sure to use as \Adonai{} so following spaces
%   work properly.
\def\Adonai{L\footnotesize{}ORD\normalsize{}}

% font and typesetting niceties
\usepackage{upquote}
\usepackage{microtype}
\setlength{\emergencystretch}{3em}


\title{{{[}TITLE{]}}}
\author{{{[}AUTHOR{]}}}

\PassOptionsToPackage{hyphens}{url}
\usepackage[hyperfootnotes=false,hidelinks,unicode]{hyperref}
\hypersetup{
    pdftitle={{{[}TITLE{]}}},
    pdfauthor={{{[}AUTHOR{]}}},
    pdflang={en-US},
    pdfcreator={SJML Paper v0.5.13},
    % pdfproducer={},
}
% for nicer linebreak placements in URLs
\usepackage{xurl}
% for underlines on links
\usepackage{href-ul}

% page numbers centered in footer
\pagestyle{plain}

% tables
\usepackage{longtable}
\usepackage{booktabs}
\usepackage{array}

% default tables are opulently tall
\let\oldlongtable\longtable
\def\longtable{\onehalfspacing \oldlongtable}

% images
\usepackage{graphicx}
\makeatletter
\def\maxwidth{\ifdim\Gin@nat@width>\linewidth\linewidth\else\Gin@nat@width\fi}
\def\maxheight{\ifdim\Gin@nat@height>\textheight\textheight\else\Gin@nat@height\fi}
\makeatother

\graphicspath{{../content}} % image paths relative to content directory
% image aspect ratios stay correct, able to use [caption](./image.png){width=2in}
\setkeys{Gin}{width=\maxwidth,height=\maxheight,keepaspectratio}

% captions
\makeatletter
%% figure captions (centered and italicized)
\setlength\abovecaptionskip{\z@}
\setlength\belowcaptionskip{-.5\baselineskip}
\long\def\@makecaption#1#2{%
	\vskip\abovecaptionskip
	\center\small\emph{#1. #2}\par
	\vskip\belowcaptionskip
}
%% table captions (why can't these be the same?!)
\patchcmd{\LT@makecaption}%
   {#1{#2: }#3}
   {#1\small\emph{#2. #3}}{}{}
\makeatother

% lists should be single-spaced
\let\oldenumerate\enumerate
\def\enumerate{\oldenumerate \singlespacing}
\let\olditemize\itemize
\def\itemize{\olditemize \singlespacing}

% tight list command required by pandoc output
\makeatletter
\providecommand{\tightlist}{%
    \ifthenelse{\equal{\the\@listdepth}{1}}
    {}
    {\vspace{-\baselineskip}}
}
\makeatother

% tighter footnote spacing
\usepackage{footnote}
\makesavenoteenv{figure}
\makesavenoteenv{longtable}
\DeclareRobustCommand{\href}[2]{#2\footnote{\url{#1}}}
\setlength{\footnotesep}{16pt}

% section headers are left-aligned and slightly larger font
\let\oldsection=\section
\renewcommand\section[1]{\oldsection{\texorpdfstring{\large\raggedright{#1}}{#1}}}
\let\oldsubsection=\subsection
\renewcommand\subsection[1]{\oldsubsection{\texorpdfstring{\large\raggedright\emph{#1}}{#1}}}
\let\oldsubsubsection=\subsubsection
\renewcommand\subsubsection[1]{\oldsubsubsection{\texorpdfstring{\raggedright\emph{#1}}{#1}}}

% standard bibliography environment taken from default pandoc template
%   except where noted (original comments removed)
\NewDocumentCommand\citeproctext{}{}
\NewDocumentCommand\citeproc{mm}{%
  \begingroup\def\citeproctext{#2}\cite{#1}\endgroup}
\makeatletter
 \let\@cite@ofmt\@firstofone
 \def\@biblabel#1{}
 \def\@cite#1#2{{#1\if@tempswa , #2\fi}}
\makeatother
\newlength{\cslhangindent}
\setlength{\cslhangindent}{1.5em}
\newlength{\csllabelwidth}
\setlength{\csllabelwidth}{3em}
\newlength{\cslitemsep} % custom separation between items
\setlength{\cslitemsep}{1em}
\newenvironment{CSLReferences}[2]
 {
\newpage % start on new page
\singlespacing % single-space it
\centerline{\underline{Bibliography}} % label it
\begin{list}{}{%
  \setlength{\itemindent}{0pt}
  \setlength{\leftmargin}{0pt}
  \setlength{\parsep}{0pt}
  \setlength{\leftmargin}{\cslhangindent}
  \setlength{\itemindent}{-1\cslhangindent}
  \setlength{\itemsep}{\cslitemsep}}} % use custom separation
 {\end{list}}
\usepackage{calc}
\newcommand{\CSLBlock}[1]{\hfill\break#1\hfill\break}
\newcommand{\CSLLeftMargin}[1]{\parbox[t]{\csllabelwidth}{\strut#1\strut}}
\newcommand{\CSLRightInline}[1]{\parbox[t]{\linewidth - \csllabelwidth}{\strut#1\strut}}
\newcommand{\CSLIndent}[1]{\hspace{\cslhangindent}#1}


% where the magic happens
\begin{document}
    
        \begin{center}

        \pagenumbering{gobble}
        \thispagestyle{empty}
        \vspace*{1in}
        \begin{singlespace}
                {{[}TITLE{]}}
                \end{singlespace}
        \vspace{2in - \baselineskip}

        by
        \vspace{2in - \baselineskip}

        {{[}AUTHOR{]}}

        \vspace{2in - \baselineskip}

        \singlespace{
            {{[}PROFESSOR{]}} \\
            {{[}MNEMONIC{]}} --- {{[}CLASS\_NAME{]}} \\
            
        }
        \end{center}
        \newpage
        \pagenumbering{arabic}
    

Nested lists!

\begin{itemize}
\tightlist
\item
  Level 1 Item 1
\item
  Level 1 Item 2
\item
  Level 1 Item 3

  \begin{itemize}
  \tightlist
  \item
    Level 2 Item 1
  \item
    Level 2 Item 2

    \begin{itemize}
    \tightlist
    \item
      Level 3 Item 1
    \end{itemize}
  \item
    Level 2 Item 3
  \item
    Level 2 Item 4
  \end{itemize}
\item
  Level 1 Item 4

  \begin{itemize}
  \tightlist
  \item
    Level 2 Item 1
  \end{itemize}
\end{itemize}

Then here is the following paragraph. Id irure irure ea dolor ex fugiat
deserunt mollit in cillum esse. Fugiat labore labore elit dolor
exercitation reprehenderit. Aute esse ullamco dolore commodo magna
reprehenderit. Mollit velit consequat aliquip dolor esse culpa ullamco
id reprehenderit enim adipisicing id reprehenderit et.

And once more, for fun, a smaller list, also in the middle of a
paragraph. Veniam pariatur duis officia sint mollit sunt Lorem sint
amet. Duis esse elit anim pariatur cupidatat enim:

\begin{enumerate}
\def\labelenumi{\arabic{enumi}.}
\tightlist
\item
  Item 1

  \begin{itemize}
  \tightlist
  \item
    Sub-item
  \end{itemize}
\item
  Item 2
\item
  Item 3
\end{enumerate}

\noindent{}And what comes after. Enim adipisicing velit cupidatat
aliquip incididunt voluptate.

Finally, some lists with longer items.

\begin{enumerate}
\def\labelenumi{\arabic{enumi}.}
\tightlist
\item
  Nulla voluptate voluptate ut cupidatat et ad do velit et id magna.
  Duis est dolore anim ex. Quis esse ut aliquip sint aliquip. Culpa enim
  nisi reprehenderit veniam cupidatat elit occaecat cupidatat aute sint
  ut. Ipsum consequat sunt aute excepteur sit culpa non esse nostrud
  veniam magna eiusmod. Ullamco est amet fugiat proident ullamco
  adipisicing aute id sit exercitation fugiat. Cupidatat voluptate
  proident in do consequat velit nulla ex consectetur. Pariatur magna in
  reprehenderit est ad reprehenderit dolor nulla culpa ex deserunt
  pariatur nisi.
\item
  Ex aliqua non ullamco eiusmod et veniam nulla quis voluptate. Ex duis
  irure deserunt adipisicing veniam officia ea quis. Sunt nisi in mollit
  ipsum est nulla. Nulla aliquip qui do irure dolor culpa aliqua
  deserunt.
\end{enumerate}

\end{document}
